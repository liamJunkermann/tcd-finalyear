\exam{Assignment 2}{Question 1}
\begin{tcolorbox}[title=Question]
  (a) [APPLICATION QUESTION] Describe how you would reliably locate emergency exit signs such as those shown the top row of images below.  Your solution must consist of a series of computer vision techniques and you must provide details of how the techniques will be applied including expected input and output for each technique.   [25 marks] 
\end{tcolorbox}

\newpage
\begin{tcolorbox}[title=Question]
  (b) [COMPARE \& CONTRAST QUESTION] Compare and contrast:
  \begin{itemize}
    \item Non-Maxima Suppression as used in first derivative edge detection. 
    \item Non-Maxima Suppression as used in the first stage of SIFT. 
    \item Non-Maxima Suppression as used in Moravec corner detection. 
    \item Non-Maxima Suppression as used in the Hough transform for circles where the radius is unknown.  
  \end{itemize}
  You must provide a list of the differences and similarities between the techniques.   Each of the differences and similarities must be clearly explained.  NOTE:  Marks will only be awarded for the detailed comparison of techniques.  No marks will be awarded for separate descriptions of the techniques                    [25 marks] 
\end{tcolorbox}

\begin{description}
  \item[1st derivative edge detection] reduces the effect lower gradients (first derivatives) might have on the result, reducing the number of double counts for an edge.
  \item[SIFT] works similarly to moravec. Since we are looking for corners we are looking for the most cornery corner of an edge, non-maxima suppression works here by look for the 
  \item[Moravec]
  
  \item[Hough] 
\end{description}