\begin{center}
\underline{\textbf{CHECKLIST}}

\underline{\textbf{Please ensure that you have submitted the following documents with your application:}}
\end{center}

\begin{table}[ht]
    \centering
    \begin{tblr}{
      width = \linewidth,
      colspec = {Q[35]Q[867]Q[38]},
      hlines,
      vlines,
    }
    \textbf{1.} & SCSS Ethical \textbf{Application Form} & Yes\\
    \textbf{2.} & {\textbf{Participant’s Information Sheet }must include the following:\\\labelitemi\hspace{\dimexpr\labelsep+0.5\tabcolsep}Declarations from Part A of the application form;\\\labelitemi\hspace{\dimexpr\labelsep+0.5\tabcolsep}Details provided to participants about how they were selected to participate;\\\labelitemi\hspace{\dimexpr\labelsep+0.5\tabcolsep}Declaration of all conflicts of interest.} & Yes\\
    \textbf{3.} & {\textbf{Participant’s Consent Form }must include the following:\\\labelitemi\hspace{\dimexpr\labelsep+0.5\tabcolsep}Declarations from Part A of the application form;\\\labelitemi\hspace{\dimexpr\labelsep+0.5\tabcolsep}Researchers'~contact details provided for counter-signature (your participant will keep one copy of the signed consent form and return a copy to you).} & Yes\\
    \textbf{4.} & {\textbf{Research Project Proposal }must include the following:\\\labelitemi\hspace{\dimexpr\labelsep+0.5\tabcolsep}You must inform the Ethics Committee \textbf{who }your intended participants are i.e. are they your work colleagues, class mates etc.\\\labelitemi\hspace{\dimexpr\labelsep+0.5\tabcolsep}How will you recruit the participants i.e. \textbf{how }do you intend asking people to take part in your research? For example, will you stand on Pearse Street asking passers-by?\\\labelitemi\hspace{\dimexpr\labelsep+0.5\tabcolsep}If your participants are under the age of 18, you must seek both parental/guardian AND child consent.} & Yes\\
    \textbf{5.} & Intended~\textbf{questionnaire}/survey/interview, protocol/screenshots/representative materials (as appropriate) & No\\
    \textbf{6.} & \textbf{URL }to intended on-line survey (as appropriate) & No
    \end{tblr}
\end{table}

\underline{\textbf{Notes on Conflict of Interest}}
\begin{enumerate}
    \item If  your intended participants are work colleagues,  you must declare a potential conflict of interest: you are taking advantage of your existing relationships in order to make progress in your research.  It is best to acknowledge this in your invitation to participants.
    \item If your research is also intended to direct commercial or other exploitation, this must be declared. For example, \textit{“Please be advised that this research is being conducted by an employee of the company that supplies the product or service which form an object of study within the research.”}
\end{enumerate}

\underline{\textbf{Notes for questionnaires and interviews}}
\begin{enumerate}
    \item If your questionnaire is paper based, you must have the following opt-out clause on the top of each page of the questionnaire: \textit{"Each question is optional. Feel free to omit a response to any question; however the researcher would be grateful if all questions are responded to."}
    \item If you questionnaire is \textbf{on-line}, the first page of your questionnaire must repeat the content of the information sheet. This must be followed by the consent form. If the participant does not agree to the consent, they must automatically be exited from the questionnaire.
    \item Each question must be \textbf{optional}.
    \item The participant must have the option to '\textbf{not submit, exit without submitting}' at the final submission point on your questionnaire.
    \item If you have open-ended questions on your questionnaire you must warn the participant against naming \textbf{third parties}: \textit{"Please do not name third parties in any open text field of the questionnaire. Any such replies will be anonymised."}
    \item You must inform your participants regarding illicit activity: "\textit{In the extremely unlikely event that illicit activity is reported I will be obliged to report it to appropriate authorities.}" 
\end{enumerate}